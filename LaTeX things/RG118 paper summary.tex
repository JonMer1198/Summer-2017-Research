\documentclass{article} 
\title{RG118 Summary} 
\author{Jonathan Mercedes Feliz} 
\begin{document} 
\maketitle{}
\section{Summary} 

\
\par A majority of galaxies with mass greater than 10$^{10}$  solar masses, contain a massive black hole in their centers. These same black holes share similar properties with its host galaxy. Relations like bulge stellar velocity disperion, bulge luminosity, and bulge mass. Yet for low/smaller BH/galaxy masses, these constraints dont hold well with BH mass smaller than 10${^6}$ $M_{\odot}$. This is one of the reasons why the amount of galaxies with less than 10$^{10}$ $M_{\odot}$ have a black hole is unknown, measuring for one in these regions is quite difficult. Knowing about these low mass galaxies with BHs are important for BH Seed formation models. AGNs within dwarf galaxies open possibilities for measuring the mass of the black hole. Being able to look at the BH activity and by also observing broad emission lines.
\\
\par Looking at one such low mass galaxy can give us some more insight for BHs within dwarf galaxies. Setting the sights on one like RG118, a dwarf disk galaxy with a mass of about 2.5x10${^9}$ $M_{\odot}$ at a redshift z=0.0243. Some evidence like Broad H-alpha emission and a nuclear x-ray point source of radiation points to BH accretion. We can then use virial theorem to calculate the mass of the BH. Using this estimation on an acceptable mass of 5x10${^4}$ $M_{\odot}$ is given for the BH within RG118. This result gives us the smallest size for a BH within a galactic nuclei. Calculating luminosity, using bolometric correction, and attaining the Eddington ratio produces similar results as AGNs within bigger systems. There are other set of possibilities for the activity seen, but those have been looked at and are not likely to be the reason. The slope/scatter of the $M_{BH}$ - $\sigma$ relationship can gain some constraints for the low mass end it, with the fraction of dwarf galaxies containing BHs.

\end{document}